\section{Wstęp}

\begin{itemize}
 \item Problem
 \begin{itemize}
  \item Analiza ruchu sieciowego
  \begin{itemize}
   \item Inofrmacje o trasowaniu, a~w~szczególności adresy sieciowe, dwóch komunikujących się ze sobą urządzeń są zawsze widoczne. Nawet jeśli treść pakietu jest zaszyfrowana, to nagłówek musi być wioczny.
   \item Przykłady (tcpdump, wireshark)
  \end{itemize}

 \end{itemize}

 \item Rozwiązanie problemu
 \item Cechy trasowania cebulowego
 \begin{itemize}
  \item Zapobiega analizie ruchu sieciowego
  \item Zapewnia anonimową, dwukierunkową komunikację, dla protokołów, które mogą używać serwisów proxy
  \item Twórcy: David M. Goldschlag, Michael G. Reed, Paul F. Syverson
  \item Używa serwerów proxy
  \item Ukrywa informacje o trasowaniu (routingu)
 \end{itemize}

 \item Opis poszczególnych rozdziałów
\end{itemize}

\subsection{Problem}
Obecnie korzystając z~Internetu, każde nasze połączenie może zostać śledzone.\foonote{Analiza ruchu sieciowego}

\subsection{Czym jest siec Tor?\protect\footnote{https://www.torproject.org/about/overview.html.en}}
\begin{itemize}
 \item Składa się z "ochotniczych" serwerów
 \item Użytkownik tworząc połączenie z~pewnym hostem, nawiązuje połączenia z~wieloma wirtualnymi tunelami\footnote{Czemu wirtualnymi?}, zamiast łączyć się bezpośrednio
 \item Pozwala na obejście cenzury, pozwalając użytkownikom na połączenie się z~blokowanym celem, lub zawartością
 \item Pozwala tworzyć serwisy Internetowe bez potrzeby ujawniania własnej lokalizacji
 \item Zapobiega analizie ruchu sieciowego
 \item Pozwala dziennikarzom na bezpieczną komunikację z~informatorami, oraz dysydentami
 \item Pracownicy mogą łączyć się bezpiecznie z~miejscem pracy
 \item Ograny ścigania używają sieci Tor do odwiedzania lub nadzoru stron sieciowych bez pozostawiania rządowego adresu IP w~logach siecowych, oraz w~celach bezpieczeństwa
 \item Im więcej osób używa sieci Tor, tym jest ona bezpieczniejsza\footnote{https://www.freehaven.net/doc/fc03/econymics.pdf}
\end{itemize}

\subsection{Analiza ruchu sieciowego\footnote{https://www.torproject.org/about/overview.html.en}}
\begin{itemize}
 \item Jest używana do określenia kto z~kim się komunikuje
 \item Przykłady skutków bycia ofiarą analizy ruchu sieciowego
 \begin{itemize}
  \item Może mieć wpływ na książeczkę czekową, jeśli witryna e-commerence używa dyskryminacji cenowej na tle państwowym, lub pochodzenia instytucji
  \item Ma wpływ na twoją pracę i~fizyczne bezpieczeństwo (ujawniając kim i~gdzie jesteś)
 \end{itemize}
 \item Jak działa analiza ruchu sieciowego?
 \begin{itemize}
  \item Przesyłany w~Internecie pakiet posiada nagłówek, oraz dane. Nagłówek nie jest zaszyfrowany, a~co za tym idzie widoczne są m.in.: adres źródłowy, adres docelowy, rozmiar pakietu, czas, itp.
  \item Inne sposoby analizy ruchu sieciowego...
 \end{itemize}
\end{itemize}

\subsection{Działanie - opis}
\begin{itemize}
 \item Rozkłada ruch pomiędzy przekaźniki
 \item Każdy przekaźnik zna tylko adres poprzedniego przekaźnika i~adres kolejnego przekaźnika
 \item Działa tylko dla protokołu TCP i~dla aplikacji wspieracjących SOCKS
 \item W~celu zwiększenia efektywności bezpieczeństwa raz na dziesięć minut tworzone jest nowe połączenie
 \item Nie ukrywa inforamcji o~konfiguracji komputera (do tego celu można użyć Tor Browser)
 
\end{itemize}

Jak opisać problem anonimowości (analizy ruchu sieciowego), wolności słowa, cenzury, prywatności?
\begin{itemize}
 \item 
\end{itemize}

Podczas korzystania z~Internetu narażeni jesteśmy na wiele ..., które ograniczają naszę podstawowe prawa osobiste\footnote{Jakiś link}. Są to np. brak anonimowości, który jest naruszany przez osoby stosujące analizę ruchu sieciowego, cenzura, czy brak wolności słowa. Na te wszystkie problemy pomaga powstała w ... roku siec Tor. Jest to grupa ochotniczych serwerów, które przekierowywują nasze połączenie. Dzięki temu nasz adres jest niewidoczny dla napastni

Jednym z~ataków na jakie można się natknąć podczas używania Inernetu jest analiza ruchu sieciowego. Proces ten polega na śledzeniu pakietów i~analizowaniu ich zawartości. Dzięki temu napastnik jest w~stanie określić lokalizację, a~nawet tożsamość komunikujących się ze sobą stron.