\section{Trasowanie Cebulowe}

\subsection{Ogólna zasada działania}
\begin{itemize}
  \item Sieć Tor składa się z~grupy węzłów pośredniczących, zwanymi Routerami Cebulowymi. Każdy z~tych węzłów utrzymywany jest przez ochotników.
  \item Sieć Tor znajduje się pomiędzy Cebulowym Proxy (klientem), a serwerem docelowym.
  \begin{enumerate}
    \item OP pobiera listę wszystkich serwerów, wraz z~ich kluczami publicznymi (kluczami cebulowymi)
    \item OP wybiera losowo kilka serwerów\footnote{ile?}, przez które będzie pośredniczony ruch
    \item OP szyfruje przesyłaną wiadomość za pomocą kluczy cebulowych w~kolejności odwrotnej niż ta w~której znajdują się węzły (najpierw ostatni w~obwodzie, a~na końcu pierwszy z~nich)
    \item OP wysyła wiadomość do pierwszego węzła w~obwodzie
    \item pierwszy węzeł odszyfrowuje wiadomość za pomocą swojego klucza prywatnego, tym samym odsłaniając adres kolejnego węzła do którego ma trafić wiadomość
    \item pierwszy węzeł wysyła wiadomość do drugiego węzła, który odsłania adres kolejnego
    \item drugi węzeł wysyła wiadomość do kolejnego
    \item proces jest powtarzany do momentu, kiedy wiadomość trafi do ostatniego węzła 
    \item ostatni węzeł odszyfrowuje wiadomość za pomocą swojego klucza prywatnego, i~wysyła ją do serwera docelowego w~oryginalnym stanie
  \end{enumerate}
  \item Podsumowanie
  \begin{itemize}
    \item 
  \end{itemize}
\end{itemize}

\begin{itemize}
  \item Sieć Tor/Trasowanie Cebulowe
  \begin{itemize}
    \item Ogólna zasada działania
    \item Tworzenie połączeń pomiędzy Routerami
    \item Struktura i opis komórki
  \end{itemize}
  \item Ukryte serwisy
  \begin{itemize}
    \item cdn...
  \end{itemize}
\end{itemize}