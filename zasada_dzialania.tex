\section{Zasada działania}

\subsection{Tor: The Second-Generation Onion Router\protect\footnote{https://svn.torproject.org/svn/projects/design-paper/tor-design.html}}
\begin{itemize}
  \item Streszczenie
  \begin{itemize}
   \item Nie wymaga modyfikacji jądra
   \item Nie wymaga specjalnych uprawnień
   \item Zapewnia kompromis pomiędzy zachowaniem anonimowości, użytecznością, oraz wydajnością
   \item Działa "w~prawdziwym Internecie"
  \end{itemize}

  \item Przegląd (zmiany od generecji 1)
  \begin{itemize}
   \item Tor to nakładka na sieć, która powstała w~celu anonimizacji aplikacji opartych na protokole TCP (przeglądanie sieci, ssh, wiadomości)
   \item Klient wybiera losowe przekaźniki w~sieci, które przekierowują ruch.
   \item Każdy przekaźnik zna tylko swojego poprzednika, oraz następnika (nie można określić całej ścieżki pakietu, nadawcy i~odbiorcy)
   \item Pakiet jest przesyłany w~komórkach o~stałej wielkości, które są rozpakowywane w~kolejnych węzłach (podobnie jak cebula, stąd trasowanie cebulowe), i~przesyłane dalej
   \item "Perfect forward secrecy"
   \begin{itemize}
    \item Inicjator negocjuje klucz sesji z~każdym sukcesywnym skokiem w~obwodzie, zamiast przesyłać wielokrotnie zaszyfrowaną wiadomość
    \item Nie potrzebna jest detekcja rejestracji ruchu
    \item Bardziej niezawodne tworzenie obwodów (inicjator wie kiedy skok zawodzi i~może rozszerzyć do nowego węzła
   \end{itemize}
   \item Separacja "czyszczenia protokołu" od anonimowości
   \begin{itemize}
    \item Tor używa protokołu SOCKS jako proxy na poziomie aplikacji, który wspiera większość aplikacji opartych na protokole TCP
    \item Opiera się na funkcjach serwerów proxy działających na poziomie aplikacji, pozwalacjących zwiększyć prywatność, takich jak Privoxy\footnote{napisac cos o Privoxy}
   \end{itemize}
   \item Możliwość udostępnienia wielu strumieni TCP, poprzez jeden zbudowany obwód (nie potrzeba negocjować wielu kluczy publicznych dla każdego połączenia, co daje większą efektywność i~bezpieczeństwo)
   \item Sprawdzanie integralności przesyłanych danych
   \item Punkty spotkania i~ukryte serwisy
  \end{itemize}
  \item Zasada działania
  \begin{itemize}
   \item Tor jest nakładką na sieć
   \item Każdy router cebulowy (ang. onion router (OR)) działa jako naromalny proces w~przestrzeni użytkownika, bez żadnych specjalnych uprawnień
   \item Każdy router cebulowy utrzymuje połączenie TLS z~każdym innym routerem cebulowym, z~użyciem kluczy emeferycznych\footnote{Co to są klucze emeferyczne?}
   \item Każdy użytkownik ma uruchomione lokalne oprogramowanie nazywane cebulowym proxy (ang. onion proxy (OP)), które pobiera katalogi, tworzy obwody, oraz utrzymuje połączenia od aplikacji użytkownika
   \item Każdy OR utrzymuje dwa klucze
   \begin{itemize}
    \item Długoterminowy klucz tożsamościowy
    \begin{itemize}
     \item Podpisywanie certyfikatów TLS
     \item Podpisywanie opisu routera OR (podsumowanie o~jego kluczach, adresach, przepustowości, polityce wyjścia, itd.)
     \item Podpisywanie katalogów (przez serwery katalogowe
    \end{itemize}

    \item Krótkoterminowy klucz cebluowy
    \begin{itemize}
     \item Używany do odszyfrowania zapytać użytkowników, aby skonfigurować obwód i wynegocjować klucze emfiryczne
    \end{itemize}
    \item Protokół TLS ustanawia krótkoterminowy klucz łącza podczas komunikacji pomiędzy OR.
    \item Krótkoterminowe klucze są okresowo i niezależnie rotowane w celu ograniczenia wpływu na ujawnienie klucza

   \end{itemize}
   
  \end{itemize}

\end{itemize}


\subsection{Komórki}
   \begin{itemize}
    \item Są jednostkami komunikacji w~sieci Tor
    \item Rozmiar komórki: 512 bajtów
    \item Składa się z~nagłówka, oraz treści
    \item Nagłówek
    \begin{itemize}
     \item CircID (2 bajty) określa do którego obwodu odnosi się komórka (wiele obwodów może być zmultipleksowanych w~jednym połączeniu TLS)
     \item Komenda(1 bajt) opisuje co należy zrobić z~treścią komórki, na podstawie komendy można określić typ komórki:
     \begin{itemize}
      \item Komórki kontrolne są interpretowane przez węzeł
      \begin{itemize}
       \item padding - używane do utrzymywania połączeń
       \item create - używane do ustanawiania nowych obwodów
       \item destroy - używane do niszczenia obwodów
      \end{itemize}

      \item Komórki przekaźnikowe przenoszą dane z~jednego do drugiego końca strumienia
      \begin{itemize}
       \item Posiadają na początku treści komórki (CircID + CMD + 11 bajtów) dodatkowy nagłówek (nagłówek przekaźnikowy)
       \item streamID (2 bajty) - identyfikator strumienia (wiele strumieni może być zmultipleksowanych w~jednym obwodzie)
       \item suma kontrolna (6 bajtów)
       \item długość treści (2 bajty)
       \item komenda "przekaźnikowa" (1 bajt)
      \end{itemize}

     \end{itemize}

    \end{itemize}

   \end{itemize}

\begin{itemize}
 \item Ogólna zasada działania + odnośniki do obrazków
 \item Używane protokoły
 \begin{itemize}
  \item Warstwy transportowej (TCP, czy UDP)
  \item Warstwy aplikacji (SOCKS)
  \item 
 \end{itemize}
 \item serwery filtrujące (Privoxy)
 \item serwery pośredniczące
 \item algorytm szyfrujący
 \item długość kluczy szyfrujących
 \item obrazki ilustrujące zasadę działania
\end{itemize}

\subsection{Ukryte serwisy}
\begin{itemize}
 \item Nie potrzeba używać firewalla
 \item Specjalna pseudodomena .onion
 \item Losowo wygenerowane adresy serwisów
\end{itemize}

\subsection{Użycie sieci Tor na systemie Linux i/lub Windows}
\begin{itemize}
 \item Kolejne kroki + output z terminala
 \item Dowód działania
\end{itemize}

\subsection{Utworzenie ukrytego serwisu w systemie Linux}
\begin{itemize}
 \item Kolejne kroki + output z terminala
 \item Dowód działania
\end{itemize}

\subsection{Przyłączenie się do serwerów pośredniczących}
\begin{itemize}
 \item Kolejne kroki + output z terminala
 \item Dowód działania
\end{itemize}

\subsection{Trasowanie Cebulowe}
