\subsection{Kalendarium}

\begin{itemize}
  \item 1995
  \begin{itemize}
    \item Początkowe dyskusje na temat trasowania cebulowego przez ONR.
  \end{itemize}
  
  \item 1996
  \begin{itemize}
   \item Pojawienie się idei użycia kluczy DH zamiast kluczy cebulowych. Zapewniałyby one bezpieczeństwo na odpowiednim poziomie, lecz użycie kluczy cebulowych było efektywniejsze.
   \item 31 Maja: Pojawienie się pierwszej formalnej publikacji i~prezentacji trasowania cebulowego "Hiding Routing Information"\footnote{https://www.onion-router.net/Publications.html\#or-infohiding} na "First Hiding Workshop". W dokumencie znajdowały się m.in. funkcje, które zostały zaimplementowane w~2. generacji, np. punkty spoktania, czy oznaczanie ataków.
   \item Wydanie prototypu na maszynie z systemem Solaris 2.5.1/2.6. Składał się on z~5 węzłów działających na jednej maszynie znajdującej się w~NRL.
   \begin{itemize}
    \item Działający proxy dla HTTP i TELNET
    \item Trwają prace nad proxy dla protokołów SMTP, oraz FTP
   \end{itemize}

   \item Rozpoczęcie prac nad generacją 1. 
   \item Lipiec: zatwierdzenie kodu 1. generacji trasowania cebulowego istniejącego już w~Maju do publicznego rozowszechnienia.
  \end{itemize}
  
  \item 1997
  \begin{itemize}
   \item Aspekty niezowodności Trasowania Cebulowego są finansowane przez DARPA w ramach High Confidence Networks Program.
   \item Projekt użycia Trasowania Cebulowego do ukrytego użycia lokalizacji telefonów komórkowych i prywatnej kontroli informacji o lokalizacji w urządzeniach śledzących lokalizację został opublikowany w Security Protocols Workshop w kwietniu.
   \item Przykłady ukrytego serwisu, oraz punktów spotkania.
   \item Slajdy ilustrujące chat IRC: \\https://www.onion-router.net/Publications.html\#old-slides
   \item Opublikowanie większości projektu na IEEE Symposium on Security and Privacy.
   \begin{itemize}
    \item Trasy o zmiennej długości (generacja 0 posiadała zdolonść do tracenia warstw przez węzły pośredniczące w sytuacji, kiedy nie było bezpośredniego połączenia z kolejnym węzłem).
    \item Separacja proxy od routera (klient nie musi ufać zdalemu proxy z trasą). 
    \item Wprowadzenie zasad wyjścia.
    \item Odseparowanie modułu kryptograficznego, zdolnego od działania na oddzielnej maszynie lub na specjalistycznym sprzęcie.
    \item Odseparowane silniki bazy danych, bez centralnego punktu awarii.
    \item ...
   \end{itemize}
  \end{itemize}
  
  \item 1998
  \begin{itemize}
   \item Utworzenie kilku rozproszonych sieci generacji 0/1 składających się z 13 węzłów na NRL, NRAD i UMD.
   \item Zbudowanie przekierowywania NRAD, działającego na Windowsie NT, przekierowującego cały ruch TCP na sieć Trasowania Cebulowego bez potrzeby użycia specjalnych proxy (wymaga użycia działającej lokalnie modyfikacji jądra).
   \item Zbudowanie proxy dla HTTP, FTP, SMTP i rlogin.
   \item Maksymalne użycie prototypu generacji 0 zostało odnotowane na lokalnym stanowisku testowym w NRL. \\Średnio wystąpiło ponad 50000 odwiedzin dziennie podczas ostatnich miesięcy, a ponad 1 milion połączeń miesięcznie.
    \item 12 grudnia: Szczytowe obciążenie wyniosło 84022 połączeń.
    \item Pod koniec roku Zero Knowledge Systems ogłosiło Freedom Network podobną do Trasowania Cebulowego. Różniła się ona tym, że działała z wykorzystaniem protokołu UDP, zamiast TCP, a z sieci mogli korzystać tylko użytkownicy płacący za subskrybcję. Węzły były utrzymywane komercyjnie, a nie przez ochotników tak jak w Trasowaniu Cebulowym. Freedom Network działało od końca 1999 do końca 2001 roku. Została ona zamknięta ze względu na zbyt małą popularność, która pozwoliłaby pokryć koszty działania sieci.
   \end{itemize}
   
  \item 1999
  \begin{itemize}
   \item Otrzymanie nagrody Alan Berman Research Publication Award przez "Anonymous Connection and Onion Routing"\footnote{https://www.onion-router.net/Publications.html\#JSAC-1998}, zawierającego szczegółowe specyfikacje Trasowania Cebulowego generacji 1.
   \item Prace nad rozwojem Trasowania Cebulowego zostają wstrzymane. \\Brak wsparcia finansowego, dyrektorzy i wydawcy opuszczają NRL, prace badawcze i analityczne nadal trwają.
  \end{itemize}
  
  \item 2000
  \begin{itemize}
   \item Styczeń: Zamknięcie prototypowej sieci generacji 0. Składała się ona z pięciu routerów cebulowych działających na maszynie Sun Ultra 2 2170, posiadającej dwa procesory o taktowaniu 167 MHz i 256MB pamięci. Przez dwa lata działania, zostało przetworzonych ponad 20 mln żądań z ponad 60 krajów. W ostatnim roku zanotowano średnio ponad 50 tys. odwiedzin dzienni. 12 grudnia 1998 roku zanotowano 84022 połączeń.
  \end{itemize}
  
  \item 2001
  \begin{itemize}
   \item Wznowiono prace nad Trasowaniem Cebulowym.
   \item Trasowanie Cebulowe jest finansowane przez DARPA w ramach programu Fault Tolerant Networks Program.
   \item Trasowanie Cebulowe generacji 1 ma zostać dokończone na tyle, aby uruchomić sieć w wersji beta, poza tym ma zostać dodana odporność na uszkodzenia, oraz zarządzanie zasobami.
   \item Trasowanie Cebulowe zostało nagrodzone nagrodą Edison Invention Award.
  \end{itemize}
  
  \item 2002
  \begin{itemize}
   \item Porzucenie kodu generacji 1, ze względu na przestarzałość.
   \item Rozpoczęcie prac nad generacją 2 Trasowania Cebulowego (Tor), na bazie kodu Mateja Pfajfara. Do 2004 roku żadna część tego kodu nie pozostała w Tor.
   \item Powody rozpoczęcia pisania Trasowania Cebulowego od nowa.
   \begin{enumerate}
    \item Wielu innych osób pracowało nad swobodnie dostępnymi filtrującymi proxy, a więc nie potrzeba tworzyć własnych w celu filtrowania strumienia danych. Zamiast tego wykorzystano Privoxy.
    \item Wraz ze wzrostem popularności zapór ogniowych, protokół SOCKS stał się na tyle wszechobecny, że prawie wszystkie aplikacje na których nam zależało stały się z nim kompatybilne. Nie trzeba pisać proxy dla każde z aplikacji.
    \item Ograniczenia eksportu dotyczące kryptografii nie są już problemem.
   \end{enumerate}
  \end{itemize}
  
  \item 2003
  \begin{itemize}
   \item Finansowanie Tor przez ONR (rozwój i wdrożenie), DARPA (zarządzanie zasobami i odporność na awarie), oraz wewnętrze fianasowanie NRL przez ONR (ukryte serwisy zdolne do przetrwania).
   \item Październik: wdrożenie i wydanie sieci Tor pod wolną i otwartą licencją MIT.
   \item Do końca 2003 roku sieć Tor miała około tuzina ochotniczych węzłów, większość w USA, oraz jeden w Niemczech.
  \end{itemize}

  \item 2004
  \begin{itemize}
   \item Wiosna: wydanie ukrytych serwisów, zbudowanie ukrytej wiki.
   \item Q4: Zakończenie finansowania przez ONR i DARPA. 
   \item Rozpoczęcie finansowania Tor przez EFF (wdrażanie i rozwój).
   \item Wewnętrzne finansowane NRL (przez ONR) ukrytych serwisów nadal trwa.
   \item Statystyki: 
   \begin{itemize}
    \item Koniec 2004 roku: Ponad 100 węzłów na trzech kontynentach.
    \item Połowa 2005 roku: Ponad 160 węzłów na pięciu kontynentach.
   \end{itemize}
  \end{itemize}

  \item 2006
  \item ...
\end{itemize}
